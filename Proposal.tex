% Setup
\documentclass{article}

% Packages
\usepackage{hyperref}
\usepackage[hyphenbreaks]{breakurl}

% Document Information
\title{Sample Selection with WaveGAN}
\author{James Hill}
\date{}

% Document
\begin{document}

\maketitle

\section{Introduction}

Generated audio has potential practical applications such as text-to-speech and digital audio music production.
\newline
\newline
A practical audio generator would need to respond to an input and generate a corresponding output: written words could produce spoken words; digital audio commands (midi) could produce musical notes as played by a musical instrument.
\newline
\newline
Generative Adversarial Networks (Goodfellow et al., 2014) are unsupervised algorithms that have produced consistently improved results for image generation.
\newline
\newline
Recently an application of GANs to unsupervised audio generation introduced the WaveGAN model (Donahue et al., 2018) which was shown to successfully capture (and generate) words from a real data distribution for small-vocabulary speech.
\newline
\newline
In an unsupervised generative model such as the initial WaveGAN there is no control on the modes of data (output) being generated.
\newline
\newline
There are now two proposed models for controlling the output of a GAN: Conditional GAN and Controllable GAN but it has not yet been demonstrated which is most capable of giving the best results.

\section{Related Work}

Generative Adversarial Networks were recently tested for audio applications for the first time by Donahue et al.
Two strategies were tested: a frequency domain strategy and a time domain strategy.
The frequency domain strategy was based on a naive application of image generating GAN methods to spectograms.
The time domain strategy was an approach that operated on the raw audio; this was the basis of the WaveGAN model.
\newline
\newline
WaveGAN was adapted from a Deep Convolutional GAN model (Radford et al., 2016) that has been shown to produce convincing generation of scenes.
\newline
\newline
Conditioning of GANs to produce controllable generated output has been previously demonstrated on image generation applications (Mirza et al., 2014).
Mirza et al. were able to succesfully control the output of a single GAN trained generative model by inserting 1-hot vectors alongside random input into the generator.
\newline
\newline
A recently proposed and tested approach is the Controllable Generative Adversarial Network (Lee et al., 2017).
CGAN uses a three network model with a classifier network introduced alongside the regular generator and discriminator.
Examples of compared output from both conditional and controllable GANs are shown in the paper but no rigorous evaluation is offered.

\section{Presentation of the Problem}

The goal of the project is to produce the best possible WaveGAN model capable of generating different outputs as controlled by the user.
There are two existing methodologies for achieving this: Conditional GAN and Controllable GAN.
An essential part of the project is to determine through experimentation whether either of these, or any other possible model, will produce the best results.

\subsection{Preliminary Testing}

Initially, a replica of the initial unsupervised WaveGAN will be trained.
It will use near identical structure and hyperparameters as the model produced by Donahue et al.
Unlike the original model it will be trained only on two of the numbers from the dataset: zero and one.
Zero and one are chosen as they are the binary numbers, have different sounds, and also differ in number of syllables.
The process of creating this initial model will allow for familiarization of the process of producing a mode from code to runtime; any difficulties encountered should be encountered and overcome within a simplified setting. 
\newline
\newline
The second model will introduce conditioning into the original model.
The new model will be similar to the first but with the introduction of an additional input layer that gives class labels to the zero and one.
This model will be trained as a simplified version of a full conditional GAN; again this only uses two of the possible numbers.
If successful, this model will prove that the concept of a conditional WaveGAN is sound.

\subsection{Full Model}

The full conditional WaveGAN model will be an expansion of the concept model previously produced.
This model will use the full ten words for the numbers from ``zero'' through ``nine''.
Ideally this model would be output at a sample rate equivalent to that used by Donahue et al. for the original WaveGAN model to allow for comparative evaluation of samples between models.
This may not be possible however if there is a requirement to use lower sample rates for the audio.

\subsection{Prototype Synthesizer}

If there is time available after the completion of the full WaveGAN model, a further model can be created in an attempt to model a prototype of a 'synthesizer'.
A synthesizer is an electronic musical instrument that artificially produces musical notes.
Synthesizers make sounds that are consciouslly artificial, transparently artificial emulations of real instruments, or fairly convincing emulations of real instruments.
Synthesizers use different models with increasing fidelity to real instruments: AM, FM, and physically modelled.
\newline
\newline
A real synthesizer would be able to produce musical notes over a range of octaves each with the 12 notes of the Western musical scale.
In order to simplify the experiment, an attempt would be made to only reproduce a simulation of the notes of a single instrument over a single scale (8 notes with 1st and last note repeated).
(If possible, a single octave (12 notes) may be trained but due to the size of the dataset a smaller range of notes would likely be more attainable for this intial experiment.)
\newline
\newline
Data for this model would be drawn from the NSynth dataset.

\section{Evaluation}

Evaluation of a Conditional WaveGAN that can generate words will be by comparing the quality of the reproduced sounds against those of the original WaveGAN.
If downsampling has taken place, then an equivalent downsampling of the generated sounds from the original WaveGAN may also take place in order that comparisons are fair.
This is acceptable as the comparison is between sets of generated samples and not against the absolute standard of real data.

\section{Potential Hazards}

The initial WaveGAN model was run for 4 days on an NVIDIA P100 GPU before converging (on sale for upwards of £6k at the time of writing).
It is possible that this level of processing power will not be available.
\newline
\newline
Firstly, the experiment can prove the concept of a conditional WaveGAN without using the full dataset used for the creation of the initial WaveGAN.
The initial dataset used words for numbers zero through nine.
A lower count of words may be chosen; two words only should be enough to prove the concept.
\newline
\newline
Secondly, in order to reduce the amount of data to be processed, the experiment may also use downsampled versions of the dataset.
Downsampling of audio may reduces the definition and will reduce the perceived quality (at least at the lower sample rates that would be needed in this case).
This will not limits the validity of the experiment and resulting model however because the perceived audio quality is not a key measurement.

\section{Datasets}

For the concept model, the dataset \textit{Speech Commands Zero Through Nine} (SC09) introduced by Donahue et al. will be used.
This is a subset of the \textit{Speech Commands Dataset} (Warden, 2017) containing only the words for numbers ``zero'' through ``nine''.
There are 1850 utterances of each word.
Each utterance is one second in length.
This results in 5.3 hours of speech.
\newline
\newline
NSynth is a large-scale dataset of musical notes (Engel et al., 2017).
It contains ~300k four-second annotated notes samples at 16kHz from ~1k harmonic music instruments.

\section{Future Work}

Further experimentation with a synthesizer would be possible by also introducing velocity as a second dimension of the conditional input (velocity relates to the hardness with which a musical note is played on an instrucment).
This would require the model to learn a further dimension of the sounds and would be prototype for ultimately more realistic synthesizers.

\section{References}

\begin{enumerate}

\item
  Donahue, Chris, McAuley, Julian, Puckette, Miller. Synthesizing Audio with Generative Adversarial Networks.
\item
  Engel, Jesse, Resnick, Cinjon, Roberts, Adam, Dieleman, Sander, Eck, Douglas, Simonyan, Karen, and Norouzi, Mohammad. Neural audio synthesis of musical notes with WaveNet autoencoders. In ICML, 2017.
\item
  Goodfellow, Ian, Pouget-Abadie, Jean, Mirza, Mehdi, Xu, Bing, Warde-Farley, David, Ozair, Sherjil, Courville, Aaron, and Bengio, Yoshua. Generative adversarial networks. In NIPS, 2014.
\item
  Lee, Minhyeok, Seok, Junhee. Controllable Generative Adversarial Network. 2017.
\item
  Radford, Alec, Metz, Luke, and Chintala, Soumith. Unsupervised representation learning with deep convolutional generative adversarial networks. In ICLR, 2016.
\item
  Warden, Pete. Speech commands dataset.
\burl{https://research.googleblog.com/2017/08/launching-speech-commands-dataset.html}, 2017.
Accessed:

\end{enumerate}

% Wrapup
\end{document}
